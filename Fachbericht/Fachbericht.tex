\documentclass[twocolumn]{revtex4}

\usepackage[utf8]{inputenc}
\usepackage[T1]{fontenc}

\usepackage{textcomp}

\usepackage[ngerman]{babel}

\usepackage{amsmath}
\usepackage{amsfonts}
\usepackage{amssymb}

\usepackage{siunitx}
\sisetup{
  output-decimal-marker={,},
  separate-uncertainty
}

\usepackage{graphicx}

\usepackage{hyperref}
\hypersetup{
  colorlinks = true,
  allcolors = {black}
}



\begin{document}

\title{Atomfallen}

\author{Christopher Deutsch}

\email[E-Mail:\,]{christopher.deutsch@uni-bonn.de}

%\affiliation{Institut, Adresse}


\begin{abstract}
%
lineare Paul-Falle, Magneto-optische Falle (max. 150 Wörter)
%
\end{abstract}

\maketitle

Viele Fortschritte der modernen Atomphysik konnte durch die Entwicklung neuer Methoden zur Isolation und Manipulation quantenmechanischer Systeme gemacht werden.
Solche Methoden umfassen unter anderem das Fangen von Atomen in sog.~Atomfallen, aber auch das Kühlen dieser Teilchen durch Laserkühlen.
Atomfallen dienen dem Einschluss von Atomen in einem endlichen Raumvolumen über längere Zeit, durch ortsabhängige Kräfte.
In diesem Sinne wäre eine Gasflasche, die ein Gas durch die Gefäßwand im Volumen einschließt, bereits als Atomfalle zu klassifizieren.
In der Atomphysik ist jedoch Fangen von Atomen ohne materielle Wände wünschenswert, da die harten Stöße mit der Gefäßwand Einfluss nimmt auf die innere Struktur des Atomgases.
Darüberhinaus steht das Gas im thermischen Kontakt mit der Umgebung, was ().
Dazu verwendet man Kräfte (?), welche einen Potentialtopf mit einer charakteristischen Tiefe $\Delta E$ erzeugen.
Oftmals wird die Energie der Teilchen durch eine Temperatur~$T$ gemäß:
\begin{align}
	E = k_\mathrm{B} \cdot T
\end{align}
angegeben.
In diesem Sinne ist die Kühlung für Atomfallen wichtig, da nur Teilchen mit einer Energie $< \Delta E$ in einer Falle gefangen werden können.
Wichtige Anwendungen:
\begin{itemize}
	\item Strukturuntersuchung von Materie bei tiefen Temperaturen (BEC, Ionenkristalle)
	\item hochauflösende Spektroskopie
	\begin{itemize}
		\item Dopplereffekt, Zeitdilatation
		\item Frequenzstandards / Atomuhren (NP 1989: Ramsay, Paul, Dehmelt)
	\end{itemize}
	\item Manipulation quantenmechanischer Systeme einzelner Teilchen
\end{itemize}

Zum Fangen von Ionen (\emph{Ionenfallen}) kann die Kraft auf geladene Teilchen im elektromagnetischen Feld ausgenutzt werden.
Tendenziell ist die Kraft auf freie Ladungen groß im Vergleich zur Kraft auf el. / mag. Dipole in inhomogenen Feldern (optische Pinzette, Magnetfalle) oder Streukraft (magneto-optische Falle).
Daher können bereits mit kleinen Betriebsspannungen hohe Fallentiefen erreicht werden.



Einführung: Was sind Atomfallen?, Fallentiefe/Potential, Energiecharakterisierung durch Temperatur, Kühlung, Anwendung von Atomfallen

Ionenfalle: Grundlagen, lin. Paul-Falle

Fallen für neutrale Atome: Streukraft, Magneto-optische Falle

Abschließender Vergleich


\begin{thebibliography}{}
\bibitem{foot}
C.\,J.\,Foot, {\it Atomic Physics} (Oxford University Press, 2005) 1st ed.

\bibitem{trapping}
C.\,E.\,Wieman, D.\,E.\,Pritchard, D.\,J.\,Wineland, {\it Atom cooling, trapping, and quantum manipulation}, Rev. Mod. Phys. \textbf{71}, 2 (1999)

\end{thebibliography}


\end{document}
