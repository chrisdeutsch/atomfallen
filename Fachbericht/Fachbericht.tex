\documentclass[twocolumn]{revtex4}

\usepackage[utf8]{inputenc}
\usepackage[T1]{fontenc}

\usepackage{textcomp}

\usepackage[ngerman]{babel}

\usepackage{amsmath}
\usepackage{amsfonts}
\usepackage{amssymb}

\usepackage{siunitx}
\sisetup{
  output-decimal-marker={,},
  separate-uncertainty
}

\usepackage{graphicx}

\usepackage{hyperref}
\hypersetup{
  colorlinks = true,
  allcolors = {black}
}



\begin{document}

\title{Atomfallen}

\author{Christopher Deutsch}

\email[E-Mail:\,]{christopher.deutsch@uni-bonn.de}

%\affiliation{Institut, Adresse}


\begin{abstract}
%
Hier sollte eine Kurzfassung (was hat der Leser zu erwarten) von maximal 150 Worten stehen.
\\
Der ganze Fachbericht darf nicht länger als wie 2\,Seiten sein. Inklusive Referenzen. ändern sie nichts an den Seitenrändern, Zeilenabständen, ...
%
\end{abstract}

\maketitle

Einführung: Was sind Atomfallen?, Fallentiefe/Potential, Energiecharakterisierung durch Temperatur, Kühlung, Anwendung von Atomfallen

Ionenfalle: Grundlagen, lin. Paul-Falle

Fallen für neutrale Atome: Streukraft, Magneto-optische Falle

Abschließender Vergleich


\begin{thebibliography}{}
\bibitem{foot}
C.\,J.\,Foot, {\it Atomic Physics} (Oxford University Press, 2005) 1st ed.

\bibitem{trapping}
C.\,E.\,Wieman, D.\,E.\,Pritchard, D.\,J.\,Wineland, {\it Atom cooling, trapping, and quantum manipulation}, Rev. Mod. Phys. \textbf{71}, 2 (1999)

\end{thebibliography}


\end{document}
