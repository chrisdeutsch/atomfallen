\documentclass[16pt]{beamer}

% Theme
\usetheme{Malmoe}
\usecolortheme{rose}
\usefonttheme[onlymath]{serif}

% Pakete
\usepackage[utf8]{inputenc}
\usepackage[german]{babel}

\author{Christopher Deutsch}
\title{Atomfallen}
\subtitle{Ein Überblick über typische Fallen}
%\logo{}
\institute{Rheinische Friedrich-Wilhelms-Universität Bonn \\ Proseminar Präsentationstechnik SS15}
\date{1. Juni 2015}
%\subject{}
%\setbeamercovered{transparent}
%\setbeamertemplate{navigation symbols}{}

\begin{document}
	\maketitle
	
	\begin{frame}
		Test
		\begin{align}
			E = \sqrt{(m c^2)^2 + p^2 c^2}
		\end{align}
		
		\begin{block}{Ein Block}
			Inhalt des Blocks
		\end{block}
		
		\begin{exampleblock}{Beispiel:}
			Eine einfache Gleichung
			\begin{align*}
				1 + 2 = 3
			\end{align*}
		\end{exampleblock}
		
		\begin{alertblock}{Pass up!}
			Achtung Achtung ich bin doof
		\end{alertblock}
			
	\end{frame}
\end{document}