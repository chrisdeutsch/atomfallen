\documentclass[12pt]{beamer}

% Pakete
\usepackage[utf8]{inputenc}
\usepackage[german]{babel}

% Theme
\usetheme{Boadilla}
\usecolortheme{rose}
\useoutertheme{infolines}
\useinnertheme{rectangles}
\usefonttheme[onlymath]{serif}

% Navigationsleiste ausschalten
\beamertemplatenavigationsymbolsempty


\author[Christopher Deutsch]
{Christopher Deutsch}

\title
{Atomfallen}

\subtitle
{Ein Überblick über typische Fallen}
%\logo{}

\institute[]
{Rheinische Friedrich-Wilhelms-Universität Bonn \\
Proseminar Präsentationstechnik SS15}

\date{1. Juni 2015}

%\setbeamercovered{transparent}
%\setbeamertemplate{navigation symbols}{}

\begin{document}
	\maketitle
	
	\begin{frame}
		\frametitle{Inhalt}
		\tableofcontents
	\end{frame}
	
	\section{Paul-Falle}
	\subsection{Formeln}
	\begin{frame}
		Test
		\begin{align}
			E = \sqrt{(m c^2)^2 + p^2 c^2}
		\end{align}
		
		\begin{block}{Ein Block}
			Inhalt des Blocks
		\end{block}
		
		\pause
		
		\begin{exampleblock}{Beispiel:}
			Eine einfache Gleichung
			\begin{align*}
				1 + 2 = 3
			\end{align*}
		\end{exampleblock}
		
		\pause
		
		\begin{alertblock}{Pass up!}
			Achtung Achtung ich bin doof
		\end{alertblock}
			
	\end{frame}
	
	\section{Magneto-optische Falle}
	\begin{frame}
		\frametitle{Magneto-optische Falle}
		\begin{columns}[t]
			\column{.5\textwidth}
				Lalala hier ist eine Column und was weiß ich nicht alles.
				
			\column{.5\textwidth}
				Lalala hier ist eine weitere Spalte
				\begin{block}{Testblock}
					Hier das ist ein Block in einer Spalte
				\end{block}
		\end{columns}
	\end{frame}
	
	\section{Laserfalle}
	\begin{frame}
		Inhalt...
	\end{frame}
	
\end{document}