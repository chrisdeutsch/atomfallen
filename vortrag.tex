\documentclass[12pt]{beamer}

% Pakete
\usepackage[utf8]{inputenc}
\usepackage[german]{babel}

% Theme
\usetheme{Boadilla}
\usecolortheme{rose}
\useoutertheme{infolines}
\useinnertheme{rectangles}
\usefonttheme[onlymath]{serif}

% Navigationsleiste ausschalten
\beamertemplatenavigationsymbolsempty


\author[Christopher Deutsch]
{Christopher Deutsch}

\title
{Atomfallen}

\subtitle
{Ein Überblick über typische Fallen}
%\logo{}

\institute[]
{Rheinische Friedrich-Wilhelms-Universität Bonn \\
Proseminar Präsentationstechnik SS15}

\date{1. Juni 2015}

%\setbeamercovered{transparent}
%\setbeamertemplate{navigation symbols}{}

\begin{document}
\maketitle

\begin{frame}{Inhalt}
	\tableofcontents
\end{frame}


\section{Motivation}

\begin{frame}{Was sind Atomfallen?}
\end{frame}

\begin{frame}{Wofür braucht man Atomfallen?}
\end{frame}

\begin{frame}{Grundlagen}
\end{frame}


\section{Ionenfallen}

\begin{frame}{Grundlagen und das Earnshaw-Theorem}
\end{frame}

\begin{frame}{Paul-Falle (mechanisches Analogon)}
\end{frame}

\begin{frame}{Paul-Falle}
\end{frame}

\begin{frame}{Penning-Falle(?)}
\end{frame}


\section{Optische Fallen}

\begin{frame}{Grundlagen: Streukraft}
\end{frame}

\begin{frame}{Grundlagen: Dopplereffekt}
\end{frame}

\begin{frame}{Grundlagen: Feinstruktur und Zeeman-Aufspaltung}
\end{frame}

\begin{frame}{Optische Melasse}
\end{frame}

\begin{frame}{Magneto-optische Falle}
\end{frame}

\begin{frame}{Dipol-Falle (opt. Pinzette) (?)}
\end{frame}


\section{Magnetfalle}

\begin{frame}{Magnetfalle}
\end{frame}


\section{Vergleich}

\begin{frame}{Fallentiefen}
\end{frame}

\end{document}