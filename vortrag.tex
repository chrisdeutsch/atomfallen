\documentclass[12pt]{beamer}

% Pakete
\usepackage[utf8]{inputenc}
\usepackage[german]{babel}

% AMS Pakete
\usepackage{amsmath}
\usepackage{amsfonts}
\usepackage{amssymb}

% Einheiten
\usepackage{siunitx}
\sisetup{
	output-decimal-marker={,},
	separate-uncertainty
}

% Grafiken
\usepackage{graphicx}


% Theme
\usetheme{Boadilla}
\usecolortheme{rose}
\useoutertheme{infolines}
\useinnertheme{rectangles}
\usefonttheme[onlymath]{serif}

% [num] Zitationen
\setbeamertemplate{bibliography item}[text]

% Navigationsleiste ausschalten
\beamertemplatenavigationsymbolsempty

\DeclareMathOperator{\divergence}{div}

\author[Christopher Deutsch]
{Christopher Deutsch}

\title
{Atomfallen}

\subtitle
{Überblick über die Funktionsweise der wichtigsten Fallen}
%\logo{}

\institute[]
{Rheinische Friedrich-Wilhelms-Universität Bonn \\
Proseminar Präsentationstechnik SS15}

\date{8. Juni 2015}

%\setbeamercovered{transparent}
%\setbeamertemplate{navigation symbols}{}

\begin{document}
\maketitle

\begin{frame}{Inhalt}
	\tableofcontents
\end{frame}


\section{Motivation}

\begin{frame}{Was sind Atomfallen?}
	\begin{itemize}
		\item Teilchen elastisch mit einer Koordinate im Raum verbunden:
		\begin{align}
		\vec{F} = - \alpha \vec{r}
		\end{align}
		Parabolische Potentiale
		\item Suspend matter without material walls
		\item Minimaler Einfluss auf die innere Struktur
		\item Minimale Heizung durch die Umgebung
		\item Beobachtung von einzelnen Atomen über längere Zeit
	\end{itemize}
\end{frame}

\begin{frame}{Wofür braucht man Atomfallen?}
	\cite{wpw}:
	\begin{itemize}
		\item BEC (niedrige Temperaturen)
		\item High-Resolution-Spectroscopy (Dopplereffekt, Zeitdilatation vermieden)
		\item Atomuhren (NP 1989 Ramsey, Paul, Dehmelt) (Auch Spektroskopie)
		\item Quantencomputer (Qubit)
		\item Quantum measurements on single atoms
		\item Quantum-state engineering
	\end{itemize}
\end{frame}

\begin{frame}{Grundlagen}
	\begin{itemize}
		\item Fallenpotential und Tiefe
		\begin{align}
			E = k_\mathrm{B} \cdot T
		\end{align}
		\item Temperatur der Teilchen (Raumtemperatur $\approx \frac{1}{40} \si{eV}$)
		\item Köhlung?
		
	\end{itemize}
\end{frame}


\section{Ionenfallen}

\begin{frame}{Grundlagen der Ionenfallen}
	\begin{itemize}
		\item Kraft auf geladene Teilchen groß (EM-WW ist stark)
		\item Große Fallentiefen (Keine Kühlung nötig) Teilchen können direkt in der Falle ionisiert werden
		\item typische Fallentiefen: $\SI{6e6}{\kelvin}$ bei Betriebsspannung \SI{500}{V}
	\end{itemize}
\end{frame}



\begin{frame}
	1. Maxwell-Gleichung für den ladungsfreien Raum (Quellenfreiheit):
	\begin{align}
	\divergence \vec{E} = 0
	\end{align}
	\begin{align}
	\int_{V} \divergence \vec{E} \, \mathrm{d}V = \oint_{S = \partial V} \vec{E} \cdot \vec{n} \, \mathrm{d}S = 0
	\end{align}
	

\end{frame}

\begin{frame}
	Fangen eines positiven Ions in einem Raumvolumen: $\vec{E} \cdot \vec{n} \stackrel{!}{<} 0$
	\begin{figure}
		\centering
		\includegraphics*[width=0.5\textwidth]{./figures/earnshaw.pdf}
	\end{figure}
		\begin{block}{Earnshaw-Theorem:}
			Ein geladenes Teilchen kann nicht durch elektrostatische Kräfte in einem Raumvolumen gefangen werden.
		\end{block}
\end{frame}

\begin{frame}{Lineare Paul-Falle ($x$-$y$-Ebene)}
	\begin{columns}[t]
		\column{0.5\textwidth}
		Quadrupol:
		\begin{figure}[h]
			\centering
			\includegraphics[width=0.8\textwidth]{./figures/lineare_paulfalle_xy_statisch.pdf}
		\end{figure}
		
		\column{0.5\textwidth}
		Quadrupol-Potential:
		\begin{align}
		\phi = \phi_0 + \frac{U_0}{2 \, r_0^2} (x^2-y^2)
		\end{align}
		\begin{figure}[h]
			\centering
			\includegraphics[width=0.95\textwidth]{./figures/sattelpotential.pdf}
		\end{figure}
	\end{columns}
\end{frame}

\begin{frame}{Lineare Paul-Falle ($x$-$y$-Ebene)}
	\begin{columns}[t]
		\column{0.5\textwidth}
		Quadrupol:
		\begin{figure}[h]
			\centering
			\includegraphics[width=0.8\textwidth]{./figures/lineare_paulfalle_xy_statisch_swapped.pdf}
		\end{figure}
		
		\column{0.5\textwidth}
		Quadrupol-Potential:
		\begin{align}
		\phi = \phi_0 - \frac{U_0}{2 \, r_0^2} (x^2-y^2)
		\end{align}
		\begin{figure}[h]
			\centering
			\includegraphics[width=0.95\textwidth]{./figures/sattelpotential2.pdf}
		\end{figure}
	\end{columns}
\end{frame}

\begin{frame}{Lineare Paul-Falle ($x$-$y$-Ebene)}
	Nutze Trägheit des Teilchens und verwende zeitlich variierendes Potential:
	\begin{align}
		\phi = \phi_0 + \frac{U_0}{2 \, r_0} \cos(\Omega t) \, (x^2-y^2)
	\end{align}
	Typische Frequenzen im Radiobereich $\nu = \mathcal{O}(\SI{10}{MHz})$
	
	Stabilität der Falle ist sehr sensitiv auf Ladung der Teilchen, RF-Frequenz, Feldeigenschaften und \textbf{Masse} der Teilchen
\end{frame}

\begin{frame}{Lineare Paul-Falle (Erweiterung auf die $z$-Achse)}
	\begin{columns}[t]
		\column{0.6\textwidth}
		\begin{itemize}
			\item 1
			\item 2
		\end{itemize}
		\column{0.4\textwidth}
			\begin{figure}[h]
				\centering
				\includegraphics[width = 1\textwidth]{./figures/lineare_paulfalle.pdf}
			\end{figure}
	\end{columns}

\end{frame}

\begin{frame}{Lineare Paul-Falle}
Axiale Kraft kleiner als Radiale:
\begin{figure}[h]
	\centering
	\includegraphics[width=0.9\textwidth]{./figures/29_laser_cooled_ion_chain.jpg}
	\caption{29 lasergekühlte Thorium-Atome ($\mathrm{Th}^{3+}$) in einer linearen Paul-Falle (Abstand etwa \SI{50}{\micro\metre}) \cite{campbell}}
\end{figure}

(Penning-Falle?)

\end{frame}

\section{Optische Fallen}

\begin{frame}{Einleitung}
	\begin{itemize}
		\item Fangen von neutralen Atomen
	\end{itemize}
\end{frame}

\begin{frame}{Grundlagen: Streukraft}
	\begin{columns}[t]
		\column{0.5\textwidth}
		\begin{itemize}
			\item Photonenimpuls:
			\begin{align}
			p = \frac{E}{c}
			\end{align}
			\item Absorption
			\item spontante Emission (Isotrop)
			\item mittlere (bremsende) Kraft
		\end{itemize}
		
		\column{0.5\textwidth}
		\begin{figure}[h]
			\centering
			\includegraphics[width=1.0\textwidth]{./figures/streukraft.pdf}
		\end{figure}
		
	\end{columns}

\end{frame}

\begin{frame}{Grundlagen: Feinstruktur und Zeeman-Aufspaltung}
\end{frame}

\begin{frame}{Dopplereffekt}
	Doppler-Effekt $k = \omega / c$:
	\begin{align}
		\Delta \omega = - k v \cos(\theta)
	\end{align}
\end{frame}

\begin{frame}{Optische Melasse}
\begin{figure}[h]
	\centering
	\includegraphics[width=\textwidth]{./figures/melasse.pdf}
\end{figure}
\end{frame}

\begin{frame}{Magneto-optische Falle}
\SI{1}{K} tief

\begin{figure}[h]
	\centering
	\includegraphics[width=0.8\textwidth]{./figures/mot.pdf}
\end{figure}
\end{frame}

\begin{frame}{Dipol-Falle (opt. Pinzette) (?)}
\end{frame}


\section{Magnetfalle}

\begin{frame}{Magnetfalle}
typische Fallentiefe: \SI{0.07}{\kelvin}
Majorana Übergang!
\end{frame}


\section{Vergleich}

\begin{frame}{Fallentiefen}
\end{frame}

\begin{frame}{Vielen Dank für die Aufmerksamkeit}
\end{frame}

\section{Literatur}

\begin{frame}{Literatur}
	\begin{thebibliography}{9}
		\bibitem{foot}
		Christopher J. Foot,
		\emph{Atomic Physics},
		Oxford University Press 2005
		
		\bibitem{campbell}
		Corey J. Campbell,
		\emph{Trapping, Laser Cooling, and Spectroscopy of Thorium IV},
		Georgia Institute of Technlogy 2011
		
		\bibitem{wpw}
		Carl E. Wieman, David E. Pritchard, David J. Wineland,
		\emph{Atom cooling, trapping, and quantum manipulation},
		Reviews of Modern Physics, Vol. 71, No.2 1999
		
	\end{thebibliography}
	
\end{frame}

\end{document}