\documentclass[12pt,xcolor=dvipsnames]{beamer}

% Pakete
\usepackage[utf8]{inputenc}
\usepackage[german]{babel}

% AMS Pakete
\usepackage{amsmath}
\usepackage{amsfonts}
\usepackage{amssymb}

\usepackage{braket}

\usepackage{multirow}

% Einheiten
\usepackage{siunitx}
\sisetup{
	output-decimal-marker={,},
	separate-uncertainty
}

% Grafiken
\usepackage{graphicx}
\usepackage{tabularx}
\newcommand{\korr}[1]{{\textcolor{Yellow}{(#1)}}}

% Theme
\usetheme{Boadilla}
\usecolortheme{rose}
\useoutertheme{infolines}
\useinnertheme{rectangles}
\usefonttheme[onlymath]{serif}

% [num] Zitationen
\setbeamertemplate{bibliography item}[text]

% Navigationsleiste ausschalten
\beamertemplatenavigationsymbolsempty

\DeclareMathOperator{\divergence}{div}

\author[Christopher Deutsch]
{Christopher Deutsch}

\title
{Atomfallen}

\subtitle
{Ein Überblick der wichtigsten Fallen}
%\logo{}

\institute[]
{Rheinische Friedrich-Wilhelms-Universität Bonn \\
Proseminar Präsentationstechnik SS15}

\date{8. Juni 2015}

%\setbeamercovered{transparent}
%\setbeamertemplate{navigation symbols}{}

\begin{document}
\maketitle

\begin{frame}{Inhalt}
	\tableofcontents
\end{frame}


\section{Motivation}

\begin{frame}{Was sind Atomfallen?}
	\begin{itemize}
		\item Einschluss von Atomen im Raumvolumen durch ortsabhängige Kräfte
		
		\item Fangen ohne materielle Wände wünschenswert:
		\begin{itemize}
			\item minimaler Einfluss auf die innere Struktur des Ensembles
			
			\item thermische Isolation von der Umgebung
		\end{itemize}

	\end{itemize}
\end{frame}

\begin{frame}{Grundlagen}
	\begin{itemize}
		\item Einschluss der Atome durch Potentialtopf:
			\begin{figure}
				\centering
				\includegraphics[width=0.6\textwidth]{./figures/fallentiefe.pdf}
			\end{figure}
		\item Energie wird charakterisiert durch Temperatur:
		\begin{align}
		E = k_\mathrm{B} \cdot T
		\end{align}
		
		\item \korr{Atomfallen und Kühlung gehen Hand in Hand}
		
	\end{itemize}

\end{frame}

\begin{frame}{Wofür braucht man Atomfallen?}
	\begin{itemize}
		\item Strukturuntersuchung von Materie bei niedrigen Temperaturen (BEC, Ionenkristalle)
		
		\item hochauflösende Spektroskopie
			\begin{itemize}
				\item Minimierung von Dopplereffekt und Zeitdilatation
				\item Frequenzstandards / Atomuhren (NP 1989 Ramsey, Paul, Dehmelt)
			\end{itemize}
		
		\item Manipulation von quantenmechanischen Systemen einzelner Teilchen (Qubit)
	\end{itemize}
\end{frame}


\section{Ionenfallen}

\begin{frame}{Grundlagen der Ionenfallen}
	\begin{itemize}
		\item Kraft auf geladene Teilchen groß (EM-WW ist stark)
		\item Große Fallentiefen (Keine Kühlung nötig) Teilchen können direkt in der Falle ionisiert werden
		\item typische Fallentiefen: $\SI{6e6}{\kelvin}$ bei Betriebsspannung \SI{500}{V}
	\end{itemize}
\end{frame}



\begin{frame}
	1. Maxwell-Gleichung für den ladungsfreien Raum (Quellenfreiheit):
	\begin{align}
	\divergence \vec{E} = 0
	\end{align}
	\begin{align}
	\int_{V} \divergence \vec{E} \, \mathrm{d}V = \oint_{S = \partial V} \vec{E} \cdot \vec{n} \, \mathrm{d}S = 0
	\end{align}
	

\end{frame}

\begin{frame}
	Fangen eines positiven Ions in einem Raumvolumen: $\vec{E} \cdot \vec{n} \stackrel{!}{<} 0$
	\begin{figure}
		\centering
		\includegraphics*[width=0.5\textwidth]{./figures/earnshaw.pdf}
	\end{figure}
		\begin{block}{Earnshaw-Theorem:}
			Ein geladenes Teilchen kann nicht durch elektrostatische Kräfte in einem Raumvolumen gefangen werden.
		\end{block}
\end{frame}

\begin{frame}{Lineare Paul-Falle ($x$-$y$-Ebene)}
	\begin{columns}[t]
		\column{0.5\textwidth}
		Quadrupol:
		\begin{figure}[h]
			\centering
			\includegraphics[width=0.8\textwidth]{./figures/lineare_paulfalle_xy_statisch.pdf}
		\end{figure}
		
		\column{0.5\textwidth}
		Quadrupol-Potential:
		\begin{align}
		\phi = \phi_0 + \frac{U_0}{2 \, r_0^2} (x^2-y^2)
		\end{align}
		\begin{figure}[h]
			\centering
			\includegraphics[width=0.95\textwidth]{./figures/sattelpotential.pdf}
		\end{figure}
	\end{columns}
\end{frame}

\begin{frame}{Lineare Paul-Falle ($x$-$y$-Ebene)}
	\begin{columns}[t]
		\column{0.5\textwidth}
		Quadrupol:
		\begin{figure}[h]
			\centering
			\includegraphics[width=0.8\textwidth]{./figures/lineare_paulfalle_xy_statisch_swapped.pdf}
		\end{figure}
		
		\column{0.5\textwidth}
		Quadrupol-Potential:
		\begin{align}
		\phi = \phi_0 - \frac{U_0}{2 \, r_0^2} (x^2-y^2)
		\end{align}
		\begin{figure}[h]
			\centering
			\includegraphics[width=0.95\textwidth]{./figures/sattelpotential2.pdf}
		\end{figure}
	\end{columns}
\end{frame}

\begin{frame}{Lineare Paul-Falle ($x$-$y$-Ebene)}
	Nutze Trägheit des Teilchens und verwende zeitlich variierendes Potential:
	\begin{align}
		\phi = \phi_0 + \frac{U_0}{2 \, r_0} \cos(\Omega t) \, (x^2-y^2)
	\end{align}
	Typische Frequenzen im Radiobereich $\nu = \mathcal{O}(\SI{10}{MHz})$
	
	Stabilität der Falle ist sehr sensitiv auf Ladung der Teilchen, RF-Frequenz, Feldeigenschaften und \textbf{Masse} der Teilchen
\end{frame}

\begin{frame}{Lineare Paul-Falle (Erweiterung auf die $z$-Achse)}
	\begin{columns}[t]
		\column{0.6\textwidth}
		\begin{itemize}
			\item 1
			\item 2
		\end{itemize}
		\column{0.4\textwidth}
			\begin{figure}[h]
				\centering
				\includegraphics[width = 1\textwidth]{./figures/lineare_paulfalle.pdf}
			\end{figure}
	\end{columns}

\end{frame}

\begin{frame}{Lineare Paul-Falle}
Axiale Kraft kleiner als Radiale:
\begin{figure}[h]
	\centering
	\includegraphics[width=0.9\textwidth]{./figures/29_laser_cooled_ion_chain.jpg}
	\caption{29 lasergekühlte Thorium-Atome ($\mathrm{Th}^{3+}$) in einer linearen Paul-Falle (Abstand etwa \SI{50}{\micro\metre}) \cite{campbell}}
\end{figure}

(Penning-Falle?)

\end{frame}

\section{Optische Fallen}

\begin{frame}{Einleitung}
	\begin{itemize}
		\item Fangen von neutralen Atomen
	\end{itemize}
\end{frame}

\begin{frame}{Grundlagen: Streukraft}
	\begin{columns}[t]
		\column{0.5\textwidth}
		\begin{itemize}
			\item Photonenimpuls:
			\begin{align}
			p = \frac{E}{c}
			\end{align}
			\item Absorption
			\item spontante Emission (Isotrop)
			\item mittlere (bremsende) Kraft
		\end{itemize}
		
		\column{0.5\textwidth}
		\begin{figure}[h]
			\centering
			\includegraphics[width=1.0\textwidth]{./figures/streukraft.pdf}
		\end{figure}
		
	\end{columns}

\end{frame}

\begin{frame}{Optische Melasse}
\begin{figure}[h]
	\centering
	\includegraphics[width=\textwidth]{./figures/melasse.pdf}
\end{figure}
\end{frame}

\begin{frame}{Grundlagen: Feinstruktur und Zeeman-Aufspaltung}
	\begin{itemize}
		\item Ein Atom hat ein magnetisches Dipolmoment $\vec{\mu}$ aufgrund des Gesamtdrehimpulses $\vec{J}$ (Zusammengesetzt aus Bahndrehimpuls $\vec{L}$ und Spin $\vec{S}$).
		\begin{align}
		\vec{\mu} = - \frac{\mu_\mathrm{B} g_J \vec{J}}{\hbar} \quad \text{unnötig}
		\end{align}
		\item Die potentielle Energie eines magnetischen Dipols im homogenen Magnetfeld ist gegeben durch:
		\begin{align}
		V = - \vec{\mu} \cdot \vec{B}
		\end{align}
		\item Im externen Magnetfeld $\vec{B}$ führt dies zu einer Aufspaltung der Energieniveaus:
		\begin{align}
		\Delta E = g_J \, \mu_\mathrm{B} \, m_J \, B
		\end{align}
	\end{itemize}
\end{frame}

\begin{frame}{Magneto-optische Falle}
\SI{1}{K} tief

\begin{figure}[h]
	\centering
	\includegraphics[width=0.8\textwidth]{./figures/mot.pdf}
\end{figure}
\end{frame}

\begin{frame}{MOT}
	\begin{itemize}
		\item Dämpfungskraft \& Rückstellkraft i.e. gedämpfter harmonischer Oszillator
		\item Magnetfeld
		\item typischer Versuchsablauf: MOT beladen durch atomischen Strahl (Zeeman-Slower), MOT wird abgeschaltet damit durch optische Melasse gekühlt wird (Zeeman-Effekt hat negativen Einfluss auf Dopplerkühlung)
		\item In einer gefüllten Falle kommt es zur ständigen Streuung des Laserlichts, was mit dem Auge/CCD sichtbar ist
		\item Gebraucht zum Laden von Dipolfallen/Magnetfallen
		\item typisches Magnetfeld \SI{0.1}{\tesla\per\metre}
		\begin{align}
		\vec{B} = B \cdot \begin{pmatrix}
		x \\ y \\ -2z
		\end{pmatrix}
		\end{align}
	\end{itemize}
\end{frame}

\begin{frame}{MOT}
\begin{columns}
\column{0.5\textwidth}

\column{0.5\textwidth}
	\begin{figure}
		\centering
		\includegraphics[width=0.95\textwidth]{./figures/mot_columbia.jpg}
		\caption{Magneto optische Falle \cite{columbia}}
	\end{figure}
\end{columns}
\end{frame}


\section{Magnetfalle}

\begin{frame}{Magnetfalle}

\begin{itemize}
	\item Betrachte Atom als Ganzes im Zustand $\ket{I J F m_F}$
	\item Zeeman-Energie: $V = g_F \mu_\mathrm{B} m_F B$
%	\item Nur vom Betrag des Magnetfeldes abhängig, da der Dipol stets mit dem Magnetfeld ausgerichtet ist. (Warum nicht bei MOT?)
\end{itemize}

typische Fallentiefe: \SI{0.07}{\kelvin}
geringer Energieabstand der Zustände im Potentialminimum führen zur Vermischung der Zustände, was zur folge hat, dass manche Atome in einen nicht durch das Magnetfeld gebundenen Zustand übergehen.
Magnetfalle in Halbleiterform (s. engl. wiki)
\end{frame}

\begin{frame}{Magnetfalle}
	\begin{columns}
	\column{0.5\textwidth}
	\begin{itemize}
		\item Starkfeldsucher: $g_F m_F < 0$
		\item Schwachfeldsucher: $g_F m_F > 0$
		\item Zum Fangen Min. oder Max. von $|\vec{B}|$ benötigt.
		\item Maximum nicht Möglich wg. $\divergence \vec{B} = 0$
	\end{itemize}
	\column{0.5\textwidth}
		\begin{figure}
			\centering
			\includegraphics{./figures/magnetfalle.pdf}
		\end{figure}
	\end{columns}
\end{frame}


\section{Vergleich der Fallen}

\begin{frame}{Vergleich der Fallen}
	
	\resizebox{\textwidth}{!}{
	{\def\arraystretch{1.5}\tabcolsep=5pt
	\begin{tabular}{l | p{4cm} | p{4cm} | p{4cm} }
		& Paul-Falle & MOT & Magnetfalle \\
		\hline
		typ. Fallentiefen: & $\sim \SI{1e6}{K}$ & $\sim \SI{1}{K}$ & $\sim \SI{10}{mK}$\\
		min. Temperatur: & $\sim \SI{1}{mK}$ & $\sim \SI{100}{\micro\kelvin}$ & $\sim \SI{10}{nK}$ \\
		&&&\\
		Vor- / Nachteile: & \textcolor{Red}{nur Ionen} & \textcolor{Green}{neutrale Atome} & \textcolor{Green}{neutrale Atome}\\
		& \textcolor{Green}{zustandsunabhängig} & \textcolor{Red}{zustandsabhängig} & \textcolor{Red}{zustandsabhängig}\\
		& & \textcolor{Green}{Kühlung durch Melasse} &\textcolor{Green}{keine untere Grenze}\\
		& \textcolor{Green}{Ionisation direkt in der Falle} & \textcolor{Green}{Einfangen von Atomen bei Raumtemperatur} & \textcolor{Red}{vorangehende Kühlstufe benötigt} \\
	\end{tabular}}}
	
\end{frame}

\begin{frame}{Vielen Dank für die Aufmerksamkeit}
\end{frame}

\section{Literatur}

\begin{frame}{Literatur}
	\begin{thebibliography}{9}
		\bibitem{foot}
		Christopher J. Foot,
		\emph{Atomic Physics},
		Oxford University Press 2005
		
		\bibitem{campbell}
		Corey J. Campbell,
		\emph{Trapping, Laser Cooling, and Spectroscopy of Thorium IV},
		Georgia Institute of Technlogy 2011
		
		\bibitem{wpw}
		Carl E. Wieman, David E. Pritchard, David J. Wineland,
		\emph{Atom cooling, trapping, and quantum manipulation},
		Reviews of Modern Physics, Vol. 71, No.2 1999
		
		\bibitem{columbia}
		Columbia University: Department of Physics,
		\url{http://physics.columbia.edu/research/atomic-molecular-optical-physics},
		(Letzter Aufruf 22. Mai 2015)
		
		
	\end{thebibliography}
	
\end{frame}

\begin{frame}{Anhang}
	\begin{figure}
		\includegraphics[width=0.5\textwidth]{./figures/ionenkristall.png}
		\caption{Ionenkristall aus lasergekühlten Berylliumatomen in einer Penning-Falle \cite{wpw}}
	\end{figure}
\end{frame}

\end{document}